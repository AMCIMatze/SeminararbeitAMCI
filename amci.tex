\documentclass[conference]{IEEEtran}
\IEEEoverridecommandlockouts


\usepackage[backend=biber]{biblatex}
\usepackage{amsmath,amssymb,amsfonts}
\usepackage{algorithmic}
\usepackage{graphicx}
\usepackage{textcomp}
\usepackage{xcolor}
\def\BibTeX{{\rm B\kern-.05em{\sc i\kern-.025em b}\kern-.08em
    T\kern-.1667em\lower.7ex\hbox{E}\kern-.125emX}}
\bibliography{amci_references}
\begin{document}

\title{Mobile Augmented Reality\\
}

\author{\IEEEauthorblockN{1\textsuperscript{st} Matthias Kerat 2\textsuperscript{nd} Mario Schlagenweith 3\textsuperscript{st} Marcel Kohnle}
\IEEEauthorblockA{\textit{Fortgeschrittene Mensch Computer Interaktion, Masterstudiengang Hochschule Aalen}\\
Aalen, Deutschland\\
matthias.kerat@studmail.htw-aalen.de, mario.schlagenweith@studmail.htw-aalen.de, marcel.kohnle@studmail.htw-aalen.de}
}

\maketitle

\begin{abstract}
This document is a model and instructions for \LaTeX.
This and the IEEEtran.cls file define the components of your paper [title, text, heads, etc.]. *CRITICAL: Do Not Use Symbols, Special Characters, Footnotes, 
or Math in Paper Title or Abstract.
\end{abstract}

\begin{IEEEkeywords}
component, formatting, style, styling, insert
\end{IEEEkeywords}

\section{Introduction}
This document is a model and instructions for \LaTeX.
Please observe the conference page limits. 



\section{AR Kernprinzipien und zentrale Begriffe}






\renewcommand*{\bibfont}{\small}
\printbibliography
\end{document}
